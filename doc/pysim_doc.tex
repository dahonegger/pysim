\documentclass[final,authoryear]{svjour3}
%\documentclass[twocolumn,final,authoryear]{svjour3}
\smartqed
\usepackage{float}
\usepackage{graphicx}
\usepackage{textcomp}
\usepackage{natbib}
\usepackage{amsmath}
\usepackage{lineno}
\journalname{ghddfgdfgd}

%\linenumbers
%\linespread{2}
\setcounter{tocdepth}{4}
\setlength{\textwidth}{170mm}
\setlength{\textheight}{230mm}


\begin{document}

%\tableofcontents
%\newpage

\title{Pysim: A Python package for bathymetry estimation using Enkf method.}

\titlerunning{Pysim v0.1}

\author{Saeed Moghimi} %, Jim Thomson}

\institute{Saeed Moghimi \at  \email{moghimis@gmail.com}
}
%\date{Received: date / Accepted: date}
% The correct dates will be entered by the editor
\maketitle
%\large

\section{Introduction}
For detail information on scientific background and idea about case study done with this package please see \cite{moghimi2015assim_syn}. 

\section{How to setup}
Here is the general steps you need to take in order to make the package working:
\begin{enumerate}
  \item Set variables in  \verb base_info.py   file.
  \item Set directories carefully  e.g. \verb base_dir,inp_dir,scr_dir. The \verb base_dir  is the main output directory.
\end{enumerate}

\section{Structure of input directory}
The \verb inp   folder has diffrent folders including:
\begin{itemize}
  \item  \verb const  : for all input information for running numerical models. Like ROMS and SWAN input files, boundary conditions, compiled excutables, bathymetry information and so on.
  \item \verb obs  : for current velocity observation. \verb sar   refers to SAR actuale data and \verb syn  refers to synthetic data for a twin test.
  \item \verb obs_swift  : for current velocity observation from a drifter.
  \item  \verb obs_wave  : for wave observation from a x-band radar or synthetic source.
\end{itemize}

\section{Structure of output directory}
The results will save in \verb base_dir which is defined in  \verb base_info.py . The software always copy the whole \verb inp  folder in order to keep each case reproduceable later on. 

For each iteration one folder as \verb run_100X will be created. each iteration folder includes several folders as:
\begin{enumerate}
  \item \verb 00_prior       : prior  bathymetry of current iteration
  \item \verb 01_bat_inp     : initial bumps ensembmle members  
  \item \verb 02_bat_adj     : final bathymetry ensebles
  \item \verb 03_mem_inp     : ROMS results for each ensmble member 
  \item \verb 04_mem_adj     : Adjusted ROMS ready to assimilate based on the time of assimilation
  \item \verb 04_swf_adj     : Drifter extracted data from ROMS runs ready for assimilation  
  \item \verb 04_wav_adj     : SWAN folder including netcdf post-processed ready for assimilation 
  \item \verb 05_assimilate  : Assimilation results directory 
  \item \verb 06_mat2prior   : Prepare posterior in netcdf fomat ready for next iterartion as new prior
  \item \verb 07_post        : folder to keep post-processing of results (plots)  
  \item \verb 08_forward     : forward ROMS run over assimilated bathymetry
  \item \verb 08_forward_swan  : forward SWAN run over assimilated bathymetry
  \item \verb scr            : a copy of scripts for this iteration as refrence.
\end{enumerate}




\bibliographystyle{spbasic}
\bibliography{References_all}

\end{document}
