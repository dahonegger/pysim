\documentclass{article}

\usepackage[dvips]{graphicx}

\begin{document}

Goal is to generate pseudo-random 2D fields having zero mean and
specified covariance.  This follows Evensen (1994) Appendix.  I won't
go through their derivation here, instead I will just concentrate on
the implementation in Matlab.

\section{1D Formulation}

I will work in 1D here, then extend to 2D later.

Using Evensen's notation, note the Matlab Inverse Fourier Transform
function ``ifft'' computes
\begin{equation}
  q_m = \frac{1}{M}\sum_{p=0}^{M-1}
  \hat{q}_p e^{-i\left(\lambda_p y_m\right)}
\end{equation}
where
\begin{eqnarray}
  y_n = n\Delta y,\\
  \lambda_p = \frac{2\pi p}{M\Delta y}.
\end{eqnarray}
This differs from the convention of Evensen, by a factor of
\begin{equation}
  \Delta\lambda = \frac{2\pi}{M\Delta y}.
\end{equation}
So computing the inverse transform (equation (A2)) can be done by
multiplying the output of ``ifft'' by $\Delta\lambda$.

Another difference is that Matlab indexes from zero frequency, which
can be confusing when it comes time to apply the Hermetian symmetry
condition (required to ensure real-valued output).  To simplify
things, I will assume an even record length, then apply
\begin{eqnarray}
  \hat{q}_1=0,\\
  \hat{q}_{M/2+1}=0,\\
  \hat{q}_m = \hat{q}_{\textrm{mod}(M-m+1,M)+1}
\end{eqnarray}
The last condition is applied for $m=2\dots M/2$.

\end{document}
